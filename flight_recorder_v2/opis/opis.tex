\documentclass[12pt,a4paper]{article}
\usepackage[T1]{fontenc}
\usepackage[polish]{babel}
\usepackage[margin=2cm]{geometry}
\usepackage[utf8]{inputenc}
\usepackage{lmodern}
\usepackage{indentfirst}
\usepackage{graphicx}
\usepackage{fancyhdr}
\usepackage{float}
\usepackage{booktabs}
\selectlanguage{polish}
\pagestyle{fancy}
\fancyhead[L]{\textbf{Flight Recorder}}
\fancyhead[R]{Jakub Serafin }
\begin{document}
\section{Opis problemu}
Problemem do rozwiązania było zaprojektowanie i stworzenie programu symulującego urządzenie zapisujące parametry samolotu podczas lotu, czyli czarnej skrzynki. 
Problemy do rozwiązania:
\begin{itemize}
\item Dane przelotu takie jak:
	\begin{itemize}
	\item prędkość
	\item wysokość
	\item wychylenia samolotu
	\item położenie
	\item przyspieszenie
	\end{itemize}
\item Przechwycenie danych,
\item Zapisanie danych,
\item Analiza danych,
\item Interfejs Graficzny
\end{itemize}
\section{Rozwiązanie}
\subsection{Dane przelotu}
W celu uzyskania danych przelotu samolotu, użyłem symulatora lotu FlightGear w wersji 2016.1.1. Posiada on możliwość zapisania logów do debuggowania. Zapisuje następujące wartości:
\begin{itemize}
\item czas od włączenia symulatora
\item położenie geograficzne - długość i szerokość
\item wysokość
\item wychylenia samolotu
\end{itemize}
Dane zostały zapisane do pliku w formacie CSV (Comma separated value). Samolot, który został zasymulowany to boeing 737.
\subsection{Przechywcenie danych}
W celu przechwycenia danych stworzyłem buffor, który ma możliwość przyjęcia danych zarówno pojedynczo jak i zbiorowo. W bufforze można ustalić, czy podczas odczytu z bufforu program ma czekać na pobranie wszystkich danych, czy wysy łać niepełne oraz ile razy ma sprawdzać, czy napłynęły nowe dane i jak często ma to robić. W celu odczytania danych z pliku stworzyłem klasę CSVReader, która wczytuje cały plik oprócz wiersza nagłówkowego. Dane wejściowe są również w prosty sposób walidowane, w celu wykrycia np. błędu zapisu do buffora bądź błędów odczytu.
\subsection{Zapisywanie danych}
Dane zapisuję za pomocą klasy FileSaver. Klasa ta zapisuje po jednej linijce do pliku o nazwie podanej w kodzie programu, jeśli taki plik już istnieje to dodaje na końcu nazwy "copy" i zapisuje dane do tak nazwanego nowego pliku.
\subsection{Analiza danych}
W celu analizy i przetworzenia danych stworzyłem klasę DataOperation, której funkcjonalności to :
\begin{itemize}
\item Zamiana czasu od początku włączenia symulatora na czas od początku lotu
\item Zamiana wysokości w stopach na wysokość w metrach
\item Zamiana współrzędnych geograficznych w czasie na odległość (w metrach) pokonaną w czasie. Są dwa różne algorytmy do wykorzystania.
\item Obliczenie całkowitej przebytej odległości.
\item Obliczenie prędkości samolotu.
\item Obliczenie przyspieszenia.
\end{itemize}
\subsection{Interfejs i prezentacja danych}
Interfejs stworzyłem przy użyciu biblioteki pyqt. Jest on umieszczony w pliku UI.py. Do stworzenia wykresów posłużyłem się biblioteką matplotlib, wykresy są umieszczone w specjalnych zakładkach, które można przesuwać i zamykać. W trakcie działania programu można zmienić dane wejściowe i stworzyć dla nich wykresy co pozwala porównać ze sobą dwa przeloty.
\end{document}